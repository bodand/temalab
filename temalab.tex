\input tikz.tex
\usetikzlibrary{intersections,arrows,angles,quotes,calc,decorations.pathmorphing,backgrounds,positioning,fit,petri}
\input epsf

\catcode`\_=11

% Primitive font set
\font\ibmserif"[fonts/IBMPlexSerif-Light.ttf]" at 11pt
\font\ibmserifBig="[fonts/IBMPlexSerif-Light.ttf]" at 21pt
\font\ibmserifbig="[fonts/IBMPlexSerif-Light.ttf]" at 17pt
\font\ibmserifbigi="[fonts/IBMPlexSerif-LightItalic.ttf]" at 17pt
\font\ibmserifb="[fonts/IBMPlexSerif-Medium.ttf]" at 11pt
\font\ibmserifi="[fonts/IBMPlexSerif-LightItalic.ttf]" at 11pt
\font\ibmserifbi="[fonts/IBMPlexSerif-MediumItalic.ttf]" at 11pt
\font\ibmmono="[fonts/IBMPlexMono-Text.ttf]" at 11pt
\font\ibmmonob="[fonts/IBMPlexMono-Bold.ttf]" at 11pt
\font\ibmmonoi="[fonts/IBMPlexMono-Italic.ttf]" at 11pt
\font\ibmmonobi="[fonts/IBMPlexMono-BoldItalic.ttf]" at 11pt

% Semantic fonts
\font\roman=\fontname\ibmserif
\font\rm=\fontname\ibmserif
\font\it=\fontname\ibmserifi
\font\bf=\fontname\ibmserifb
\font\fnt_title=\fontname\ibmserifBig
\font\fnt_code=\fontname\ibmmono
\font\fnt_subtitle=\fontname\ibmserifbigi
\font\fnt_section=\fontname\ibmserifbig
\font\fnt_subsection=\fontname\ibmserifbigi
\font\fnt_figure=\fontname\ibmserifb

% Font setter macros
\def\title#1{\fnt_title #1}
\def\subtitle#1{\fnt_subtitle #1}
\def\bold#1{{\bf #1}}
\def\ital#1{{\it #1}}
\def\itbo#1{\ibmserifbi #1}

% coloring
% Color support
% Code is mostly stolen as-is from https://tex.stackexchange.com/a/207307

\newif\ifpdfmode

\begingroup\expandafter\expandafter\expandafter\endgroup
\expandafter\ifx\csname pdfoutput\endcsname\relax
\else
  \ifnum\pdfoutput>0 %
    \expandafter\expandafter\expandafter\pdfmodetrue
  \fi
\fi

\edef\colordef#1to#2r#3g#4b{%
  \ifpdfmode
		\expandafter\xdef#1{#2 #2 #3 rg #1 #2 #3 RG}
  \else
		\expandafter\xdef#1{rgb #2 #3 #4}
  \fi
}

\edef\currentcolor{%
  \ifpdfmode
    0 g 0 G%
  \else
    gray 0%
  \fi
}

\edef\resetcolor{%
  \ifpdfmode
    \noexpand\pdfcolorstackpop
  \else
    \special{color pop}%
  \fi
}

\colordef\colorblue to 0r 0g 1b
\colordef\colorred to 0.6r 0g 0.5b

\begingroup\expandafter\expandafter\expandafter\endgroup
\expandafter\ifx\csname pdfcolorstack\endcsname\relax
  \ifpdfmode
    \def\pdfcolorstackpush{\pdfliteral{\currentcolor}}%
    \let\pdfcolorstackpop\pdfcolorstackpush
  \fi
\else
  \chardef\colorstack=0 %
  \def\pdfcolorstackpush{%
    \pdfcolorstack\colorstackcnt push{\currentcolor}%
  }%
  \def\pdfcolorstackpop{%
    \pdfcolorstack\colorstackcnt pop\relax%
  }%
\fi

\edef\color#1{%
  \begingroup\noexpand\expandafter\noexpand\expandafter\noexpand\expandafter\endgroup
  \noexpand\expandafter\noexpand\ifx\noexpand\csname color#1\noexpand\endcsname\relax
    \def\noexpand\currentcolor{#1}%
  \noexpand\else
    \noexpand\expandafter\let\noexpand\expandafter\noexpand\currentcolor
      \noexpand\csname color#1\noexpand\endcsname
  \noexpand\fi
  \ifpdfmode
    \noexpand\pdfcolorstackpush
  \else
    \special{color push \noexpand\currentcolor}%
  \fi
  \aftergroup\noexpand\resetcolor
}

\newbox\coloringbox
\def\ascolor[#1]#2{%
	\setbox\coloringbox=\hbox{\begingroup\color{#1}#2\endgroup}%
	\box\coloringbox%
}


% Color scheme using bodand-coloring.tex to define code schemes
% requires that bodand-coloring.tex is included before this

\colordef\colorvar 		to 0.549019607843137r 0.423529411764706g 0.243137254901961b
\colordef\colorfn 		to 0.219607843137255r 0.376470588235294g 0.749019607843137b
\colordef\colortype		to 0.596078431372549r 0.329411764705882g 0.945098039215686b
\colordef\colorlitstr to 0.345098039215686r 0.462745098039216g 0.223529411764706b
\colordef\colorlitnum to 0r 								0.443137254901961g 0.592156862745098b
\colordef\colorop			to 0r 								0g								 0b



\countdef\ex_code_num=130
\ex_code_num=0

\newbox\codetitlebox
\newdimen\codetitlebox_w
\def\code[#1]{%
	\vskip.42cm
	\advance\ex_code_num by 1
	\def\type{\ascolor[type]}
	\def\fn{\ascolor[fn]}
	\def\var{\ascolor[var]}
	\def\op{\ascolor[op]}
	\def\litstr{\ascolor[litstr]}
	\def\litnum{\ascolor[litnum]}
	\indent\hbox{%
		\hss%
		\vbox{
			\hbox{\fnt_code \the\ex_code_num. \hskip-7pt Példa #1}
			\hrule{}
		}
		\hss%
	}
	\vskip3pt
	\fnt_code
}
\def\cline#1{\line{\indent #1\hss}}
\def\endcode{
\vfil
\roman
\vskip.42cm}

\countdef\section_num=128
\countdef\subsection_num=129
\section_num=0
\subsection_num=0

\dimendef\section_around=128
\dimendef\subsection_around=128
\section_around=16pt
\subsection_around=11pt

\def\section[#1]{
	\vfil\break
	\advance\section_num by 1
	{\fnt_section \noindent\the\section_num. #1}
	\subsection_num=0
	\expandafter\vskip\the\section_around
	\relax
}

\def\subsection[#1]{
	\expandafter\vskip\the\subsection_around
	\advance\subsection_num by 1
	{\fnt_subsection \noindent\the\section_num.\the\subsection_num. #1}
	\expandafter\vskip\the\subsection_around
	\relax
}

\countdef\figure_num=131
\figure_num=0

\def\figuredesc#1{%
	\vskip3pt
	\advance\figure_num by 1
	\vbox{\fnt_figure \hbox to\hsize{\hss\the\figure_num. Ábra #1\hss}\vskip.314cm}
	\relax
}

% Footer
%\footline=\hbox{\centerline{\folio}}
\footline={\vbox{%
	\ifnum\folio=1%
		{\hss}%
	\else%
		\centerline{$-$\folio$-$}%
	\fi
}}

% Utilities
\def\titlepage#1#2by#3date#4{
  \vbox{
    \vskip4.5cm
    \centerline{\title{#1}}
  }
  \vskip.3cm
  \centerline{\subtitle{#2}}
  \vskip.3cm
  \centerline{#4}
  \vskip1cm
  \centerline{#3}
  \vfil\break
}

\catcode`\ˇ=8
\roman
 

\hyphenation{%
    A
    A-lap-ve-tő-en
    Az
    C
    C--ben
    E-zek
    En-nek
    Er-ről
    Ez
    Hasz-ná-la-tá-ban
    I-lyen
    IPC
    IT
    Lin-da
    Nincs
    U-tób-bi
    a
    a-da-to-kat
    a-da-tok
    a-dat-bá-zi-sá-nak
    a-dat-bá-zis
    a-dat-bá-zist
    a-dott
    a-lap-ve-tő
    a-láb-bi-ak-ban
    a-me-lyek-ben
    a-mely
    a-mely-be
    a-mi
    a-zo-kat
    a-zok
    an-nak
    az
    be
    biz-to-sí-tott
    biz-to-sí-tá-sá-val
    biz-to-sít-ja
    bold
    bő-veb-ben
    bő-vít-mény-ként
    com-mu-ni-ca-ti-on
    csak
    cso-ma-got
    csu-pán
    cél-ja
    cím-kép-ző
    de-fi-ni-ál
    de-fi-ni-ál-ja
    de-fi-ni-ál-ni
    de-fi-ni-ált
    diszk-re
    e-gye-sé-vel
    e-gyes
    e-gyéb
    e-gész
    e-le-mek
    e-le-mé-nek
    e-lem
    e-lem-re
    e-mel-lett
    e-rős
    e-se-tek-ben
    egy
    egy-más-ba
    egy-sze-rű
    egy-sze-rű-sé-ge
    egy-sé-get
    el
    el-ké-szí-tett
    el-ké-szí-té-se
    el-sőd-le-ges
    el-ér-he-tő-ek
    end-code
    esz-köz-ként
    ext-ra
    fej-léc-fájlt
    fel-a-da-ta
    fel-a-dat
    for-má-tu-ma
    fu-tás-ban
    fu-tás-i-de-jű
    fu-tó
    func-ti-on
    funk-ci-o-na-li-tá-sát
    fut-ta-tó
    függ-vé-nye-ket
    függ-vé-nyek
    függ-vé-nyek-ben
    függ-vény-hí-vás
    függ-vényt
    fő
    ga-ran-tál-ja
    gaz-da
    gaz-da-nyelv-ben
    glo-bá-li-san
    ha-son-lít
    hasz-ná-la-tos
    hasz-ná-la-tá-hoz
    hasz-nál-ha-tó
    hasz-nál-ni
    hasz-nált
    hat
    hogy
    i-gé-nyel-nek
    in-dí-tott
    in-ter-fé-szét
    is
    is-mer-te-té-sé-re
    kap-cso-la-tok
    ke-re-sett
    ke-re-sés
    ke-rül
    ke-ze-li
    ke-zel-he-tő
    kell
    ki-hagy-ha-tó
    kom-mu-ni-ká-ci-ót
    kon-zisz-ten-ci-a
    konk-rét
    kér-dő-jel
    kö-szön-he-tő-en
    kö-vet-ke-zők-ben
    kö-vet-ve
    kö-zött
    kör-nye-zet-re
    kör-nye-zet-ről
    kü-lön
    le-gy-sze-rűbb
    le-he-tő-ség
    le-he-tős-gé-ről
    le-hes-sen
    le-het-sé-ges
    le-kér-de-zé-se
    le-kér-de-zé-sek-nél
    le-í-rá-sa
    lesz
    lé-te-zik
    lét-re-ho-zá-sá-ra
    ma-gá-ban
    match
    meg
    meg-fe-le-lő
    meg-fe-le-lő-en
    meg-ha-tá-ro-zott
    meg-va-ló-sí-tá-sá-hoz
    meg-va-ló-sí-tá-sá-ra
    mi-att
    min-ta--il-lesz-té-si
    mint
    mo-dult
    má-sik
    még
    ne-vű
    nem
    nyelv
    nyelv-be
    o-lyan
    o-pe-ra-tív
    o-pe-rá-tor
    o-pe-rá-tor-hoz
    ol-va-só
    pat-tern
    pri-mi-tív
    pro-cessz--kö-zi
    pro-cessz-ből
    pro-cesszek
    prog-ra-mo-zá-si
    rész-hal-ma-zuk
    sa-ját
    se
    sor
    space
    sza-bad
    sze-man-ti-kát
    szerint
    szin-tak-ti-ka-i
    szink-ro-ni-zá-ci-ós
    szink-ro-ni-zá-ci-ót
    szá-mok
    szó
    szö-ve-gek
    szük-sé-ges
    szük-ség
    ta-lál-ha-tó
    tart-sa
    tel-jes
    to-váb-bi
    tu-laj-don-ság-nak
    tup-le
    tá-rol-ja
    tá-rolt
    tár-ban
    tár-gya-lá-sá-nál
    tér
    tér-be
    töb-bi
    tör-té-nik
    tört
    u-gyan-ab-ba
    u-tán
    va-ló-sí-tá-nak
    vagy
    van
    vfil
    vi-szont
    vál-to-zó-ba
    vé-gez-he-tő
    ágya-zá-sá-ra
    ál-tal
    áll-hat-nak
    é-pül
    ér-ke-zik
    ér-tel-me-zett
    ér-tel-me-zé-sé-hez
    ér-té-ke
    ér-té-kek
    ér-ték
    és
    így
    ír-ni
    ösz-szes
}


\def\in{\bold{in}\space}
\def\inp{\bold{inp}\space}
\def\rd{\bold{rd}\space}
\def\rdp{\bold{rdp}\space}
\def\out{\bold{out}\space}
\def\eval{\bold{eval}\space}

\titlepage{Témalaboratórium jegyzőkönyv}%
{Linda nyelv megvalósításához memóriarezidens elosztott adatbázis}%
by{Bodor András}%
date{2023/2024 Tanév 1. Félév}

\section[Feladatleírás]

A feladat a Linda nevű nyelv megvalósításához szükséges adatbázis elkészítése.
Ennek az adatbázisnak az elsődleges feladata, hogy a futó processzek között a tárolt elemek a szinkronizációját karban tartsa, erős konzisztencia biztosításával.
Az adatbázis csak egy-egy konkrét lefutásban szükséges, hogy tárolja az adatokat, így nem kell diszkre írni, csupán operatív tárban kezelni azokat. 
A tárolt adatok formátuma egyszerű {\it n}-tuple ({\it n}-es), amelyekben csak primitív értékek, mint egész-, vagy tört számok vagy szövegek állhatnak.
Nincs lehetőség az egyes tárolt elemek között kapcsolatok létrehozására, se {\it n}-tuple értékek egymásba ágyazására. 
Az értékek lekérdezése csak egyesével egy egyszerű minta-illesztési ({\it pattern match}) szemantikát követve végezhető el.

\subsection[A Linda nyelv]

A feladat fő célja, hogy az elkészített adatbázist a Linda nyelv megvalósítására lehessen használni.
Ennek megfelelően az alábbiakban a Linda nyelv ismertetésére kerül sor, ami után az adatbázis interfészét lehetséges definiálni.

A Linda nyelv egy egyszerű bővítményként kezelhető a legegyszerűbb leírása szerint. 
Alapvetően egy másik, magában is teljes programozási nyelvbe épül be, amelybe hat extra szabad függvényt (\ital{free function}) definiál.
Ezek a függvények a gazdanyelvben globálisan elérhetőek, és nem igényelnek külön fejlécfájlt (\ital{header}), csomagot (\ital{package}), modult vagy egyéb a választott nyelv által definiált egységet, amely definiálja az adott függvényeket.
A következőkben a Linda nyelv megvalósításához használt gazda nyelv a C programozási nyelv.

A függvények használatához egyes esetekben szükség van még egy szintaktikai elemre. 
Ez a \ital{? operátor}, amely az adatbázis lekérdezéseknél használatos elem, amivel a keresett {\it n}-tuple egy elemének értéke kihagyható.
Ilyen esetekben a többi érték szerint történik a keresés, és azok által meghatározott \ital{n}-tuple megfelelő elemének értéke a változóba kerül.
Használatában hasonlít a C-ben található címképző operátorhoz (\ital{\char`\&}), viszont csak az adatbázist olvasó függvényekben értelmezett.

\code[A kérdőjel operátor]
\line{\indent\type{int} \var{x};\hfil}
\line{\indent\fn{in}(\litstr{"string"}, \op{?}\var{x});\hfil}
\line{\indent\fn{printf}(\litstr{"x: \char`\%d"}, \var{x});\hfil}
\endcode

A szintaktika mellett a függvények értelmezéséhez szükség van egy futásidejű környezetre, amely az adott függvények funkcionalitását biztosítja.
Erről a futtató környezetről garantálja, hogy létezik egy olyan tuple tér (\ital{tuple space}), amelyre az összes adott processzből indított Linda függvényhívás, ugyanabba a tuple térbe érkezik.
Ennek a tulajdonságnak köszönhetően a Linda nyelv által biztosított függvények 1) egyszerű processz-közi kommunikációt (\ital{interprocess communication}, IPC) valósítanak meg a saját processzből indított további processzek között;
 és 2) részhalmazuk szinkronizációs eszközként is használható. 

\vskip.2cm
A Linda nyelv által definiált függvények listája:
\vskip.1cm

\indent\hbox to.4cm{\out\hss}
\par A függvény a paramétereiből képzett \ital{n}-tuple értéket az adatbázisba tölti.

\vskip.1cm
\indent\hbox to.4cm{\in\hss}
\par Kétféle működésre képes.

Amennyiben minden érték megadásra került a paramétereiben, tehát nincs benne ? operátor, akkor a paramétereiből képzett \ital{n}-tuple értékét kiveszi a tuple térből.
Ha ez jelenleg nem található meg benne, akkor a processz várakozni kezd, addig, amíg ez bele nem kerül, majd kiveszi azt.

A másik lehetőség, amikor paraméterezett tuple értéket kérdezünk le.
Ilyenkor valahány elem helyére egy változó ? operátorral képzett kifejezése kerül. 
Ebben az esetben egy olyan tuple értéket vesz ki az adatbázisból, amelyik a konkrétan megadott értékeknek megfelel, és a nem specifikált elemek helyére pedig a talált tuple megfelelő értéke kerül.
Ha nem található a keresésnek megfelelő tuple, akkor ebben az esetben is várakozik, amíg bele nem kerül a tuple térbe egy megfelelő tuple.

\vskip.1cm
\indent\hbox to.4cm{\rd\hss}
\par Az \in függvényhez hasonló működésű függvény, a különbség, hogy ha talált a tuple térben megfelelő tuple értéket, akkor visszatér, de nem veszi ki azt az adatbázisból.
Emiatt más processzek is olvashatják, anélkül, hogy valakinek ismételten bele kellene raknia a tuple térbe azt.

\vskip.1cm
\indent\hbox to.4cm{\inp\hss}
\par Megpróbálja az \in függvény szemantikájának megfelelően megkeresni a tuple térben a paramétereiből képzett tuple értéket, de azonnal visszatér logikailag hamis értékkel ha ez nem sikerül.

\vskip.1cm
\indent\hbox to.4cm{\rdp\hss}
\par Megpróbálja az \rd függvény szemantikájának megfelelően megkeresni a tuple térben a paramétereiből képzett tuple értéket, de azonnal visszatér logikailag hamis értékkel ha ez nem sikerül.

\vskip.1cm
\indent\hbox to.4cm{\eval\hss}
\par Az \out függvényhez hasonlóan, az \eval is a tuple térbe helyez elemeket, azonban ha egyes paramétereinek függvények vannak megadva, akkor azok nem a lokális processzben kerülnek kiértékelésre.
Ilyen esetekben a kiértékelés egy-egy másik processzen történik, ahonnan csak a végeredmény érkezik vissza, majd a megadott értékekből, és a visszakapott számítások eredményéből képzett \ital{n}-tuple kerül az adatbázisba.

\subsection[Az adatbázis]

Az előzőek alapján a megvalósítandó adatbázis képességei a következőek.

Képesnek kell lennie bármilyen \ital{n}-re eltárolnia \ital{n}-tuple értékeket, azt visszakereshető módon.

Minden tuple elem értéke egy primitív érték.
Egy primitív érték vagy egy string, vagy egy egész szám, ami fizikai memóriában 16, 32 vagy 64 bites, vagy egy lebegőpontos szám, ami egyszeres (\ital{single}, C \ital{float} típusával megfeleltethető) vagy kétszeres (\ital{double}) pontossággal rendelkezik.

Adott processzek között képesnek kell lennie konzisztens állapotban tartania az adatbázist, miszerint az egyes processzeknek mindig friss adatot kell tudnia olvasnia, ha egy másik processz írt vagy törölt valamit.

Egyes processzeknek tudnia kell hatékonyan várakozni az adatbázison, amíg az ő keresési feltételüknek megfelelő adat nem érkezik az adatbázisba.
Ezt anélkül lehessen megoldani, hogy folyamatosan az adatbázist kell figyelni, kapjon valamilyen formában értesítést a processz a közös tuple tér változásáról.

\section[Megvalósítás]

A megvalósítás LindaDB néven C++20-ban történt, több helyen kihasználva a nyelv modern lehetőségeit.
Az egyes létrehozott adatszerkezetek, sok helyet teljes támogatást biztosítanak általános használatra, biztosítanak olyan műveleteket, amelyekre az egyes szabványos függvényeknek van szüksége.
Például, az alább tárgyalt {\ibmmono \ascolor[type]{ldb::data::chunked_list}<class \ascolor[type]{T}, \ascolor[type]{std::size_t} \ascolor[var]{ChunkSize}>} iterátorait lehetséges a szabványos iterátorok helyén használni.

\subsection[Adattárolás]

Az adatokat egy saját készítésű adatszerkezet tárolja.
Alább látható egy ábra arról, hogy nagyvonalakban, hogy néz ki memóriában az adatszerkezet.
Ez a szerkezet biztosítja, hogy mind az iterátorok mindig érvényesek maradnak, akár hozzáfűznek vagy törölnek az adatok közül, mind az adatokra képzett iterátorok összehasonlíthatóak. 
Ennek megfelelően a \ital{LegacyBidirectionalIterator} nevesített követelményt (\ital{named requirement}) teljesítik, hiszen lehetséges előre és hátra is léptetni őket.
Az \ital{LegacyRandomAccessIterator} nevesített követelményt nem tudják teljesíteni, mert egy adott index eléréséhez sükség van, hogy minden egyes szegmensen végiglépkedve kiszámoljuk, hogy melyik az adott index alatt elhelyezkedő elem.
Hasonlóan egy iterátorhoz való $n$ hozzáadása esetén, lehetséges, hogy át kell lépni a következő szegmensbe, ahol pedig lehetséges hogy nincs elég elem, hogy a megfelelő számot előre lépjünk.
Ennek megfelelően, lehetséges, hogy egy iterátor indexelésnél az egész listát be kell járni, ami sérti a szemantikai követelményeit a \ital{LegacyRandomAccessIterator} nevesített követelménynek, miszerint konstans időben lehetséges ezt megtenni.

\vskip.5cm
\input chunked_list.tikz
\figuredesc{Szegmenselt lista}
\vskip.5cm

Az adatok alapvetően egy vektorban tárolt pointereken keresztül elérhető memória szegmensekben (\ital{chunk}) vannak tárolva.
Egy-egy ilyen szegmens mindig statikus méretű, amikor megtelik egy új szegmens kerül lefoglalásra, amikor minden elem törlődik
belőle, akkor a szegmens is törlődik.
A szegmens, elemszámát, és hogy melyik mezője szabad, vagy foglalt, bitenként tárolja egy, a meghatározott számú mező tárolására elegendő, de lehető legkisebb, fordítási időben kiválasztott egész típus bitjeiben.
Egyes mezők ellenőrzése egy bitmaszkolási művelet, míg az elemszámot egy \ital{popcount} művelettel végzi.

A szegmenseket egy {\ibmmono std::vector<std::unique_ptr<\ital{szegmens típus}>>} adattagban tárolja.
Ennek köszönhetően, amikor hozzáfűzni kell az adatbázishoz, akkor egyszerűen csak a legutolsó szabad szegmensbe beszúrunk, vagy ha az már nincs ilyen, akkor szimplán felveszünk még egy szegmenst és abba szúrjuk be az elemet.
Mivel az egyes vektor elemei nem maguk a szegmensek, hanem csak a szegmenseket birtokló mutatók, így ha beszúrás, vagy bármely más művelet miatt, a vektor alatti tömböt újra kell foglalni, és így átkerül máshova, az iterátorok nem invalidálódnak.
Ez a tulajdonság hasonló egy egyszerű láncolt listához, viszont jobb memória kihasználtsággal rendelkezik az adatszerkezet, egy listához képest.

Vegyünk egy egyszerű, egyirányba láncolt listát, 8 byteos mutatókkal. Az adat, amit tárolunk 8 byte, összesen 32 darab van belőle.
Így a teljes memóriaigénye a listának byteban: $$ 32 * (8 + 8) = 512 $$

Ezzel szemben a szegmenselt lista, ugyan ezen adatokkal, 8 elemű szegmensekkel, szintén byteban: $$ 24 + 8 + 8 * 4 + (1 + 8 + 8 + 8 * 8) * 4 = 396 $$
A számításban a 24 byte az elsődleges vektor adminisztratív kültsége (8 byte mutató a tömbre, 8 byte méret és 8 byte kapacitás), míg a zárojelezett rész egy szegmens. 
Eltároljuk még, az utolsó olyan szegmenst, amelyikből utoljára töröltünk, ez egy index, így 8 byte.
A következő $8 * 2$ a vektor tömbje, amiben két pointer van.
Ebben található a legkisebb méretű olyan szám, aminek a bitszáma elég, hogy $n$ elemet tároljon egy $n$ elemű szegmensben.
Ez jelenleg egy {\ibmmono \ascolor[type]{unsigned char}}, ami 1 byte.
A következő kettő 8 byte a birtokló szegmensel lista objektumra mutató mutató, és a szegmens indexének száma, a $8 * 8$ pedig a 8 darab 8 byte méretű tárolt elem.

Ez a tulajdonság elromlik, ha sok közel üres szegmens marad a rendszerben, amit nem tudunk törölni, mert különben sérülne, hogy az iterátorok mindig érvényesek maradnak.
Ennek elkerülése végett, minden törlésnél az indexét annak a szegmensnek, amiből töröltünk eltároljuk, így a következő beszúrás abba az szegmensbe kerül.
Ezzel elérhető, hogy a beszúrás ideje konstans maradjon (hiszen nem kell végignézni a teljes tartalmat, hogy üres helyet találjunk), illetve azt is elkerüljük, hogy az egyes szegmensek alacson töltöttséggel üzemeljenek.
Utóbbi annyira fontos, hogy már csupán 50\char`\%-os kihasználtság esetén, az egyszerű lista minden esetben jobb memóriakihasználtsággal rendelkezik.
Az alább látható grafikonon, 512 elemet próbálunk eltárolni, növekvő szegmens mérettel, egy egyszerű listával és két szegmenselt listával, ahol az egyikben csak 50\char`\%\space\space az egyes szegmensek kihasználtsága.
Az X tengely a szegmensek mérete, 16-tól, 512-ig lineárisan növekszik.

\epsfxsize=15cm
\epsfbox{bad_chunking.eps}
\figuredesc{Rossz hatékonyságú szegmenselt lista}

Mivel a szegmensek ismerik a saját tartalmazójukat, így iteráció esetén, mindig a magukon való túlindexelés esetén tudják továbbítani a kérést, így a fő vektort tartalmazó objektum továbbíthatja a következő szegmensnek.
Emellett még a mindenkori indexüket is ismerik a fő vektorban, így két iterátor esetén, konstans időben eldőnthető, hogy melyik van előbb, mint a másik, pont, mint egy tömbre mutató pointerek esetén.

Ezekkel a tulajdonságokkal megvalósítható hatékonyan a listához hozzáadás (vagy a végére, vagy a közepére, ahol van hely), a törlés (egy bit átírása, és a visszaadott elem move konstruktorának hívása), és bár nem teljesen éri el a egy vektor iterálását, egy átlagos láncolt szerkezetnél hatékonyabban, és cache barátiabban iterálható.
A teljes iteráció nyilván olyankor kell, amikor keresünk az adatbázisban.
Viszont ez rendes, diszk rezidens adatbázisokban is az elsődleges lassulás oka, a teljes tábla végigolvasása.
Mivel az adatszerkezetünk képes kényelmesen kezelni kötött adatokat--jelen esetben iterátorokat--hiszen soha nem invalidálja még érvényes adat iterátorát, így feltárul a lehetőség, hogy indexeket építsünk az elsődleges adatszerkezet főlé, ezzel megkönnyítve a keresését.

\subsection[Adatbázis indexek]

Az előző szekcióban leírásra került, hogy milyen adatszerkezetben kerülnek tárolásra az adatok.
Mivel ebben a szerkezetben való keresés nem hatékony, hiszen mindig egy teljes végigolvasást igényelne, így, a gyakorlati használatban is ismert adatbáziskezelőkhöz hasonlóan, a Linda nyelv megvalósítás ezen adatbázisa is indexeket vezet be az adatok gyorsabb elérésére.

Diszk rezidens adatbázisok elsődlegesen egy B-fa valamelyen variációjával (B+-fa, B*-fa, UB-fa, stb.) szokták (vagy valamilyen hash alapű adatszerkezettel) az indexelést megvalósítani, hiszen egy blokkos háttértáron való tároláshoz a blokkos szervezése nagyon kényelmes.
Operatív tárban azonban sokkal hatékonyabb, ha nem egy B-fa variánst alkalmazunk, bár a konstans blokk alapú elérést biztosítják, viszont az operatív tárban való véletlen címektre ugrást elkerülendő, valamilyen bináris keresőfa használata, amivel lehetséges kevessebb mutató követéssel elérni egyes adatokat.

Az egyszerű bináris keresőfák problémája, hogy mivel nem rendezik át magukat semmilyen esetben, így egyes bemenetek esetén a mélységük $n$-el lesz arányos. 
Legrosszabb esetben, például egy szigorúan monoton növekvő bemenet esetén, egy egyszerű listává romlik vissza a hatékonyságuk.
Erre a problémára léteznek önrendező bináris keresőfák, amelyek minden bemenet után végrehajtanak bizonyos fa forgatásokat, így garantálva, hogy a fa mélysége csak $\log{n}$ mértékben nő.
Ennek megvalósítására több algoritmus létezik; például az AVL-fa, és a piros-fekete fa két gyakran használt alternatíva.

Jelenleg AVL-fa alapú megvalósítást használunk, mivel az AVL fák szigorúbb rendezettséget biztosítanak a piros-fekete fákhoz képest, átlagosan lassabb beszúrás ellenében.
Ennek megfelelően, a fa magassága átlagosan alacsonyabb egy AVL-fa esetében, így a keresés áltagos ideje jobb.

A probléma azonban az egyszerű bináris keresőfákkal, hogy a memóriakihasználtságuk nem hatékony, minden elemhez szükség van a két gyerek elemre mutató mutató eltárolására, ami 8 byteos mutatók esetén minden elemhez 16 byte overheadet tárol.
Emellett iteratív megvalósítások esetén, amik nagy mennyiségű adat tárolása edetén szükgégesen adódnak, egy-egy gyereket szülőre mutató mutatóval is el kell látani, külünben nem lehetséges egy részfából a visszalépés.
A jelenleg megvalósított fa szerkezet is iteratív, így az adott rendszer stack mérete nem korlátozza a tárolt adatok méretét.
Erre az alapköltségre ráépítenek még az önrendező fa algoritmusok, amelyeknek az extra memóriaigényükkel számolni kell, amiben az adatokat tárolják, amivel a rendezést megvalósítják.
A használt AVL-fák esetében ez egyszerűen 1 byte, mivel a jelenlegi részva rendezettségét tárolandó 3 érték eltárolható benne.
Mikrooptimalizált megvalósítások megpróbálhatják a jelenleg használt 1 byteot a valójában szükséges 2 bitre csökkenteni, akár a pointerek alignment értékéből következő szabad bitekbe beleírva, de ez nagyon nem szép, és sok nagyon elővigyázatos memória elérést igényel, amivel az egyszerű memória címzések is extra műveleteket igényelnének, így ezzel a lehetőséggel nem él a jelenlegi implementáció. 

Ezt a memória pazarlást elkerülendő, az előzőekben említett nem szép és igazából minimális spórolást elérő lehetőséggel ellentétben, a T-fa egy olyan speciális fa adatszerkezet, amelyik az egyes csúcspontokban nem egy-egy elemet tárol, hanem egy rendezett tömböt.
A rendezett tömb első eleménél kisebb elemek találhatóak a bal gyerekben, míg a legutolsó eleménél nagyobbak találhatóak a jobb gyerekben, így megmarad a bináris keresőfa tulajdonság. 
Ezzel a megvalósítással egy-egy csúcshoz tartozó mutatók overheadje kisebb az adat méretéhez képest (hiszen amíg túl nagy az overhead, addig az tömb mérete növelhető, így közel tetszőleges memóriahatékonyság elérhető).
A T-fa algoritmus azonban nem definiálja, hogy milyen önrendező algoritmust használ az a fa szerkezet, amelybe beépül.
Ennek megfelelően, az eddigi AVL-fa szerkezetbe csak annyi átírás eszközölendő, hogy képes legyen kezelni mind egy, mind több elemű hasznos adatot. 
Ezt a hasznos adat tárolás elabsztrahálásával egy {\ibmmono avl2_tree<Payload>} sablon bevezetésével lehetséges, így elérhető, hogy az AVL-fa megvalósítás megfelelő hasznos adatot tároló sablon paraméter esetén, akár egyszerű AVL-faként, akár T-faként viselkedjen.
Megjegyzendő, hogy a tényleges definíció az alábbi, míg az előbb közölt a magyarázatot szolgálja.

\code[Az általános AVL-fa szerkezet]
\cline{template<class \type{K},}
\cline{\hskip2.12cm class \type V,}
\cline{\hskip2.12cm \type{std::size_t} \var{Chunking} = \litnum{0},}
\cline{\hskip2.12cm payload \type{Payload} = }
\cline{\hskip4.42cm          typename \type{ldb::index::tree::payload_dispatcher}<}
\cline{\hskip6.63cm 										\type{K}, \type{V}, \var{Chunking}>::type>}
\cline{struct \type{avl2_tree} \char`\{...\char`\};}
\endcode

Ez a szerkezet már sokkal memória hatékonyabb egy egyszerű AVL-fánál, amellett, hogy mivel több elem fér egy csúcsba, így a fa mélysége is kisebb, bár csak egy konstans faktorral.
Azonban a jelenlegi felhasználás mellett a kulcsok, ami szerint tároljuk az adatokat, nem biztosítanak egyediséget. 
Például a következő három tuple teljességükben páronként különbözőek, viszont az egyes elemeik közül semelyik sem biztosít egyediséget, így nem lehet rájuk egyedi indexet építeni.

$$\vbox{\halign{\hss(#, &#, &#)\hss\cr
	1 & "alma" & 41\cr
	1 & "körte" & 42\cr
	2 & "alma" & 42\cr}}$$

Ennek következtében, minden épített indexben lehetséges, hogy ugyan az a kulcs többször is szerepelhet.
Ha minden T-fa cellába, csak egy kulcs-érték párt tárolunk el, akkor lehetséges, hogy a fa nagy része ugyanazzal a kulcs értékkel lesz feltöltve, és mindegyik különböző értékeket tárol.
Viszont egy bináris keresőfa, még ha egy komplexebb megvalósításban, mint pl. egy AVL-, vagy T-fa, akkor is a kulcs szerint rendezett, így az értékek nem definiált sorrendben lesznek a fában.
Emiatt akárhányszor keresnénk egy nem egyedi kulcsra, az összes olyan csúcspontot be kellene járni, amibe belekerült legalább egy keresett kulcsérték.

Erre a megoldás, hogy minden T-fa cella, egy-egy skaláris kulcs és érték helyett, egy kulcshoz értékek egy vektorát tárolja.
Ezt az vektort pedig rendezve elérhető, hogy a minden keresés $\logˇ2{nˇk}+\logˇ2{nˇe(k)}$ idővel arányosan végrehajtható, ahol az $nˇk$ a kulcsok száma, $nˇe(k)$ pedig az értékek száma adott $k$ kulcs esetén.
Ennek a közelítőleges memória ábrája látható alább, amiből következik az elnevezése, miszerint ezt az adatszerkezetet \ital{csillingelő-fának} neveztem el.

Fejlesztési lehetőség lehet valamilyen dinamikusan változó szerkezet, amely nagy $nˇe(k)$ esetén a rendezett vektort is lecseréli egy T-fa szerkezetre, így nem lesz szükség nagyon nagy összefüggő memória részekre, és megsprórolható a beszúrások esetén az esetleges tömb teljes másolása. 
Itt a tárolt adatok mindig egyediek, így a csillingelő-fa által bevezetett indirekció megspórolható.
Ennek tényleges hasznosságát ki kellene mérni.

<<csillingelő fa ábra>>

Ezzel az adatszerkezettel, mint indexel, már leírható a Linda nyelv megvalósításához készített adatbázis jelenlegi állapota.
Az előző szubszekcióban leírt szegmenselt lista, ami az elsődleges adattárolást valósítja meg, ami köré kerülnek a csillingelő-fa indexek.
Jelenleg, mivel a tárolt \ital{n}-tuple értékek esetén, a \ital{n} értéke lehet tetszőlegesen nagy (nyilván a memória limitációin belül), így statikus darabszámú indexel nem lehetséges megoldani, hogy minden elemre lehessen megvalósítani indexelt elérést.
Ennek megfelelően, mivel az indexek karbantartása is plusz erőforrásokat és időt venne igénybe, amire még a dinamikusan szervezett indexek dinamicitásának karbantartási igénye is ráépül, a jelenlegi megvalósítás nem él ezzel a lehetőséggel.
Helyette statikusan kettő darab indexet épít, az első két elemre, és ezeket használja a keresések gyorsított elérésére.
Ezt a gyorsítási próbálkozást lehetséges kijátszani, így ha a keresett tuple első két eleméből mind kettő ? operátorral képzett kifejezés, és így nem hajtható rajta keresés végre.
Ilyen esetekben egy teljes végigolvasásra kényszerül az adatbázis, így ezek a keresések jelentősen lasabbak azoknál, amelyek nem rendelkeznek ezzel a problémával.

Jelenleg a keresés egy elég primitív algoritmus szerint hajtódik végre, lekérdezésoptimalizáció nem történik.
Az első olyan index szerint történik a keresés, amelyik értéke konkrétan meg van adva; ennek következtében a keresések legtöbbje az első indexű indexet használja.
Itt a lekérdezésoptimalizácira számos lehetőség adódik, például az egyes kulcsokhoz tartozó vektorok mérete az adott index szerint, amellett, hogy milyen magas a fa; ezekkel már becslés szinten eldönthető, hogy melyik érdemesebb irány a keresésre.
Ezek nem lettek implementálva.
Végül alább látható egy ábra a jelenlegi teljes adatbázis állapotáról.

\subsection[Konzisztencia]

Az adatbázis a konzisztenciát úgy biztosítja, hogy minden beszúrás esetén a többi futó processzt értesíti, hogy beszúrás történt, így azok is fel tudják dolgozni, és beszúrni magukba azt.
Hasonlóan a törlés esetén is, értesítődnek a többi processzek.

Ehhez egy típus kitörölt (\ital{type erased}) függvény típuson ({\ibmmono std::function<...>}) keresztül történik a megvalósítás, így esetleges más kommuikációs technológia esetén is könnyedén lehetséges az adatbázis integrálása.
Ez a függvény egy másik függvényt ad vissza. 
Erre azért van szükség, hogy az aszinkron kommunikáció alapú megvalósítás esetén, az lokális adatbázis tudja a saját magán elvégzendő beszúrást vagy törlést végrehajtani, amíg a kommunikáció történik.
Az első függvény csak elindítja a kommunikációt, de azt aszinkron módon, és a visszaadott függvény az, amelyik addig várakozik, amelyik az be nem fejeződik.
A kettő között pedig lehet az előbb említett lokális feladatot végrehajtani. 

\bye
